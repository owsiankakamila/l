\documentclass[a4paper,12pt]{article}
\usepackage[T1]{fontenc}
\usepackage[latin2]{inputenc}
\usepackage[polish]{babel}

\usepackage{amsthm}
\usepackage{times}
\usepackage{anysize}

\marginsize{1.5cm}{1.5cm}{1.5cm}{1.5cm}
\sloppy 

\theoremstyle{definition}
\newtheorem{df}{Definicja}


\begin{document}
	Istnieje scis�y zwi �azek mi�edzy rozk�adem macierzy � $A$ na macierze$ L$ i$ U$ a metod �a eliminacji Gaussa.
	Mozna wykaza � c,� ze elementy kolejnych kolumn macierzy � $L$ s �a r�wne wsp�czynnikom przez kt�re mnozone
	s �a w kolejnych krokach wiersze uk�adu r�wna � n celem dokonania eliminacji niewiadomych w odpo- �
	wiednich kolumnach. Natomiast macierz $U$ jest r�wna macierzy tr�jk �atnej uzyskanej w eliminacji Gaussa.
	
	$$
	[A|b]=
	\left[ \begin{array}{cccc}
	2 & 2 & 4 & 4 \\
	1 & 2 & 2 & 4 \\
	1 & 4 & 1 & 1 \\
	\end{array} \right] =
	\left[ \begin{array}{ccrrc}
	2 & 2 & 4 & 4 \\
	1 & 2 & 2 & 4 \\
	1 & 4 & -1 & -1 \\
	\end{array} \right] =
	\left[ \begin{array}{ccrr}
	2 & 2 & 4 & 4 \\
	1 & 2 & 2 & 4 \\
	1 & 4 & 1 & 1 \\
	\end{array} \right]	$$
	
$$
	L = \left[ \begin{array}{ccc}
2 & 2 & 4  \\
1 & 2 & 2  \\
1 & 4 & 1  \\
\end{array} \right]
\qquad
U=\left[ \begin{array}{ccrr}
2 & 2 & 4 & 4 \\
1 & 2 & 2 & 4 \\
1 & 4 & -1 & -1 \\
\end{array} \right]
$$
	


	
\end{document}