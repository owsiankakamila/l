\documentclass[a4paper,12pt]{book}      
\usepackage[utf8]{inputenc} 
\usepackage[T1]{fontenc}
\usepackage{times}
\usepackage{amssymb}
\usepackage{amsthm}
\usepackage[polish]{babel}

\sloppy


\begin{document}     

\section*{Zastosowania geometryczne całki oznaczonej} 


\paragraph{Obliczanie długości łuku:}
Jeżeli krzywa wyznaczona jest równaniem $y=f(x)$, przy czym \mbox{funkcja}
$f(x)$ ma w przedziale $[a,b]$ ciągłą pochodną, to {\em długość łuku} w tym
przedziale wyraża się wzorem: $$L = \int_{a}^{b} \sqrt{\strut 1+ [  f(x) ]^{2}} dx$$


\paragraph{Obliczanie objętości bryły obrotowej:}
Jeżeli $...$ jest funkcją ciągłą i nieujemną na na przedziale $...$, to
objętość bryły obrotowej powstałej z obrotu wokół osi $...$ linii o równaniu
$...$, gdzie $...$, wyraża się wzorem:
$$V = \pi \int_{a}^{b}  f^{2}(x) dx$$


\paragraph{Obliczanie pola powierzchni bryły obrotowej:}
Jeżeli $...$ jest funkcją ciągłą i nieujemną na przedziale $...$ i ma w
tym przedziale ciągłą pochodną, to pole powierzchni bryły obrotowej powstałej z
obrotu wokół osi $...$ linii o równaniu  $...$, gdzie $...$, wyraża się
wzorem:
$$S = 2\pi \int_{a}^{b}  f(x)\sqrt{\strut 1+ [  f(x) ]^{2}} dx$$

\end{document}
